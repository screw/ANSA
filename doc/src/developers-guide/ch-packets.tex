\chapter{Packet API}
\label{cha:packet-api}

\section{Overview}

The representation of packets is an essential modeling support for
communication network simulation. Applications and communication protocols
construct, deconstruct, encapsulate, fragment, aggregate, and manipulate
packets in many ways. In order to ease the implementation of said
behavioral patterns, INET provides a feature-rich C++ packet API.

\tbf{Packets}.
The packet API primarily provides a general purpose packet data structure
that is capable of representing application packets, \protocol{TCP}
segments, \protocol{IP} datagrams, \protocol{Ethernet} frames,
\protocol{IEEE 802.11} frames, etc. The packet data structure allows
efficient storage, duplication, sharing, encapsulation, aggregation,
fragmentation, serialization, and data representation selection. The packet
data structure consists of two layers built on one another. The upper layer
deals with packets, and the lower layer deals with alternative data
representations.

\tbf{Buffers.}
The packet API, despite its name, does not only provide the packet data
structure but several other functionality. For example, communication
protocols often need to enqueue data for sending (e.g., TCP), or buffer
received data for reassembly (e.g., IP) or for reordering (e.g., IEEE
802.11). These services are provided as separate data structures on top of
the lower layer mentioned above.

\tbf{Tags}.
Communication between protocols inside network nodes often require passing
around meta information along with packets. To this end, packets are also
capable of carrying meta information called tags. Tags are simple C++
classes which can be either attached to the whole packet or to a specific
region.

\tbf{Dissectors.}
Understanding what's inside a packet is another important and often used
functionality. Protocol specific packet dissectors allow analyzing the
contents of packets deeply. For example, packet filtering is based on
packet dissection, it allows the visualization of certain packets easing
model understanding.

\tbf{Packet printers.}
During model development, packets often need to be printed in a human
readable form. INET provides a built in packet printer which is
automatically used by the OMNeT++ runtime user interface to display packets
in the packet log window. At the network level, the packet log is very
similar to what is usually provided by network protocol analyzers such as
Wireshark.

\begin{note}
Code fragments in this chapter have been somewhat simplified for brevity.
Some \ttt{const} modifiers and \ttt{const} casts have been omitted,
setting chunk fields have been omitted, etc.
\end{note}

\section{Representing Chunks}

The packet data structure is a compound data structure that builds on top
of another set of data structures called chunks. The chunk data structures
provide several alternatives to represent a piece of data. Chunks can be
simple or compound if they are built using other chunks.

Communication protocols and applications may define their own chunks or use
already existing ones. User defined chunks are most often genereted by the
OMNeT++ MSG compiler as a subclass of \cppclass{FieldsChunk}. It's also
possible to write a user defined chunk from scratch.

INET provides the following built-in chunks:

\begin{itemize}
    \item   repeated byte or bit chunk (\cppclass{ByteCountChunk}, \cppclass{BitCountChunk})
    \item   raw bytes or bits chunk (\cppclass{BytesChunk}, \cppclass{BitsChunk})
    \item   ordered sequence of chunks (\cppclass{SequenceChunk})
    \item   slice of another chunk designated by offset and length (\cppclass{SliceChunk})
    \item   many protocol specific field based chunks (\cppclass{FieldsChunk} subclasses)
\end{itemize}

Applications and communication protocols most often construct simple chunks
to represent application data and protocol headers. The following examples
demonstrate the construction of various simple chunks.

\cppsnippet{ChunkConstructionExample}{Chunk construction example}

In general, chunks must be constructed with a call to \cppclass{makeShared}
instead of the standard \cppclass{new} C++ operator. The special
construction mechanism is required for the efficient sharing of chunks
among packets using C++ shared pointers.

Packets most often contain several chunks inserted by different protocols
as packets are passed through the protocol layers.

The most common way of forming such a compound chunk to represent packet
contents is concatenation.

\cppsnippet{ChunkConcatenationExample}{Chunk concatenation example}

Protocols often need to slice data, for example to provide fragmentation,
which is also directly supported by the chunk API.

\cppsnippet{ChunkSlicingExample}{Chunk slicing example}

In order to avoid cluttered data representation due to slicing, the chunk
API provides automatic merging for consecutive chunk slices.

\cppsnippet{ChunkMergingExample}{Chunk merging example}

Alternative representations can be easily converted into one another using
automatic serialization as a common ground.

\cppsnippet{ChunkConversionExample}{Chunk conversion example}

\msgsnippet{UdpHeaderDefinitionExample}{UDP header definition example}

\section{Representing Packets}

The packet data structure uses a single chunk data structure to represent
its contents. The contents may be as simple as raw bytes
(\cppclass{BytesChunk}) but most likely it will be the concatenation
(\cppclass{SequenceChunk}) of various protocol specific headers (e.g.,
\cppclass{FieldsChunk} subclasses) and application data (e.g.,
\cppclass{ByteCountChunk}).

Packets can be created by both applications and communication protocols. As
packets are passed down through the protocol layers in the sender node, new
protocol specific headers and trailers are inserted during processing.

\cppsnippet{PacketConstructionExample}{Packet construction example}

In order to facilitate processing by communication protocols, packets are
split into three parts: headers, data, and trailers. As packets are passed
up through the protocol layers in the receiver node, protocol specific
headers and trailers are popped during processing.

\cppsnippet{PacketProcessingExample}{Packet processing example}

\ifdraft TODO
\section{Representing Network Signals}

\fi

\cppsnippet{SignalConstructionExample}{Signal construction example}

\section{Representing Transmission Errors}

An essential part of communication network simulation is the understanding
of protocol behavior in the presence of errors. The packet API provides
several alternatives for representing errors. The alternatives range from
simple but computationally cheap to accurate but computationally expensive
solutions.

\begin{itemize}
    \item   mark erroneous packets (simple)
    \item   mark erroneous chunks (good compromise)
    \item   change bits in raw chunks (accurate)
\end{itemize}

\cppsnippet{ErrorRepresentationExample}{Error representation example}

The physical layer models support the above mentioned different error
representations via configurable parameters. Higher layer protocols detect
errors by chechking the error bit on packets and chunks, and by standard
CRC mechanisms.

\section{Tagging Packets}

Several protocols process a packet while the packet is being passed around
within a network node. The packet often needs to carry some meta
information to support this processing. The most trivial meta information
example is the outermost protocol of the packet which cannot be
unambigously identified just by looking at the raw data.

\cppsnippet{PacketTaggingExample}{Packet tagging example}


\section{Tagging Data Regions}

%TODO explain how INET packet representation allows tagging arbitrary regions of packets with meta information

\cppsnippet{RegionTaggingExample}{Region tagging example}

\section{Dissecting Packets}

\cppsnippet{PacketDissectorCallbackInterface}{Packet dissector callback interface}

\cppsnippet{PacketDissectionExample}{Packet dissection example}

\section{Filtering Packets}

\cppsnippet{PacketFilteringExample}{Packet filtering example}

\section{Printing Packets}

\cppsnippet{PacketPrintingExample}{Packet printing example}

\ifdraft TODO
\section{Recording PCAP}

TODO recorded into a PCAP file for further processing with 3rd party tools
\fi

%%%% Local Variables:
%%% mode: latex
%%% TeX-master: "usman"
%%% End:

