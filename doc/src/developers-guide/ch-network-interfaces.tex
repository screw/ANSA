\ifdraft TODO

\chapter{Network Interafces}
\label{cha:network-interfaces}

\section{Overview}

TODO

\section{The Interface Table}

The \nedtype{InterfaceTable} module holds one of the key data structures in
the INET Framework: information about the network interfaces in the host.
The interface table module does not send or receive messages; other modules
access it using standard C++ member function calls.

\ifdraft TODO
\subsection{Accessing the Interface Table}

If a module wants to work with the interface table, first it needs to obtain a
pointer to it. This can be done with the help of the
\cppclass{InterfaceTableAccess} utility class:

\begin{cpp}
IInterfaceTable *ift = InterfaceTableAccess().get();
\end{cpp}

\cppclass{InterfaceTableAccess} requires the interface table module to be a
direct child of the host and be called \ttt{"interfaceTable"} in order to
be able to find it. The \ffunc{get()} method never returns \ttt{NULL}: if
it cannot find the interface table module or cannot cast it to the
appropriate C++ type (\cppclass{IInterfaceTable}), it throws an exception
and stop the simulation with an error message.

For completeness, \cppclass{InterfaceTableAccess} also has a
\ffunc{getIfExists()} method which can be used if the code does not require
the presence of the interface table. This method returns \ttt{NULL} if the
interface table cannot be found.

Note that the returned C++ type is \cppclass{IInterfaceTable}; the initial
"\ttt{I}" stands for "interface". \cppclass{IInterfaceTable} is an abstract
class interface that \cppclass{InterfaceTable} implements. Using the abstract
class interface allows one to transparently replace the interface table with
another implementation, without the need for any change or even
recompilation of the INET Framework.
\fi

\subsection{Interface Entries}

Interfaces in the interface table are represented with the
\cppclass{InterfaceEntry} class. \cppclass{IInterfaceTable} provides member
functions for adding, removing, enumerating and looking up interfaces.

Interfaces have unique names and interface IDs; either can be used to look up
an interface (IDs are naturally more efficient). Interface IDs are invariant to
the addition and removal of other interfaces.

Data stored by an interface entry include:

\begin{itemize}
  \item \textit{name} and \textit{interface ID} (as described above)
  \item \textit{MTU}: Maximum Transmission Unit, e.g. 1500 on Ethernet
  \item several flags:
    \begin{itemize}
      \item \textit{down}: current state (up or down)
      \item \textit{broadcast}: whether the interface supports broadcast
      \item \textit{multicast} whether the interface supports multicast
      \item \textit{pointToPoint}: whether the interface is point-to-point link
      \item \textit{loopback}: whether the interface is a loopback interface
    \end{itemize}
  \item \textit{datarate} in bit/s
  \item \textit{link-layer address} (for now, only IEEE 802 MAC addresses are supported)
  \item \textit{network-layer gate index}: which gate of the network layer within the host the NIC is connected to
  \item \textit{host gate IDs}: the IDs of the input and output gate of the host the NIC is connected to
\end{itemize}

\tbf{Extensibility}. You have probably noticed that the above list does not
contain data such as the IPv4 or IPv6 address of the interface. Such
information is not part of \cppclass{InterfaceEntry} because we do not want
\nedtype{InterfaceTable} to depend on either the IPv4 or the IPv6 protocol
implementation; we want both to be optional, and we want
\nedtype{InterfaceTable} to be able to support possibly other network
protocols as well.

Thus, extra data items are added to \cppclass{InterfaceEntry} via
extension. Two kinds of extensions are envisioned: extension by the link
layer (i.e. the NIC), and extension by the network layer protocol:

\begin{itemize}

\item \tbf{NICs} can extend interface entries via C++ class inheritance, that is, by
simply subclassing \cppclass{InterfaceEntry} and adding extra data and
functions. This is possible because NICs create and register entries in
\nedtype{InterfaceTable}, so in their code one can just write
\ttt{new MyExtendedInterfaceEntry()} instead of \ttt{new InterfaceEntry()}.

\item \textbf{Network layer protocols} cannot add data via subclassing, so
composition has to be used. \cppclass{InterfaceEntry} contains pointers to
network-layer specific data structures. For example, there are pointers to
IPv4 specific data, and IPv6 specific data. These objects can be accessed with
the following \cppclass{InterfaceEntry} member functions: \ffunc{ipv4Data()},
\ffunc{ipv6Data()}, and \ffunc{getGenericNetworkProtocolData()}.
They return pointers of the types \cppclass{Ipv4InterfaceData},
\cppclass{Ipv6InterfaceData}, and \cppclass{GenericNetworkProtocolInterfaceData},
respectively. For illustration, \cppclass{Ipv4InterfaceData} is installed
onto the interface entries by the \nedtype{Ipv4RoutingTable} module, and it
contains data such as the IP address of the interface, the netmask, link
metric for routing, and IP multicast addresses associated with the
interface. A protocol data pointer will be \ttt{NULL} if the corresponding
network protocol is not used in the simulation; for example, in IPv4
simulations only \ffunc{ipv4Data()} will return a non-\ttt{NULL} value.


\end{itemize}


\subsection{Interface Registration}

Interfaces are registered dynamically in the initialization phase by modules
that represent network interface cards (NICs). The INET Framework makes use
of the multi-stage initialization feature of OMNeT++, and interface registration takes
place in the first stage (i.e. stage \ttt{INITSTAGE\_LINK\_LAYER}).

Example code that performs interface registration:

\begin{cpp}
void PPP::initialize(int stage)
{
    if (stage == INITSTAGE_LINK_LAYER) {
        ...
        interfaceEntry = registerInterface(datarate);
    ...
}

InterfaceEntry *PPP::registerInterface(double datarate)
{
    InterfaceEntry *e = new InterfaceEntry(this);

    // interface name: NIC module's name without special characters ([])
    e->setName(OPP_Global::stripnonalnum(getParentModule()->getFullName()).c_str());

    // data rate
    e->setDatarate(datarate);

    // generate a link-layer address to be used as interface token for IPv6
    InterfaceToken token(0, simulation.getUniqueNumber(), 64);
    e->setInterfaceToken(token);

    // set MTU from module parameter of similar name
    e->setMtu(par("mtu"));

    // capabilities
    e->setMulticast(true);
    e->setPointToPoint(true);

    // add
    IInterfaceTable *ift = findModuleFromPar<IInterfaceTable>(par("interfaceTableModule"), this);
    ift->addInterface(e);

    return e;
}
\end{cpp}

\ifdraft TODO
\subsection{Interface Change Notifications}

\nedtype{InterfaceTable} has a change notification mechanism built in, with
the granularity of interface entries.

Clients that wish to be notified when something changes in
\nedtype{InterfaceTable} can subscribe to the following notification
categories in the host's \nedtype{NotificationBoard}:

\begin{itemize}
  \item \tbf{\ttt{NF\_INTERFACE\_CREATED}}: an interface entry has been
    created and added to the interface table
  \item \tbf{\ttt{NF\_INTERFACE\_DELETED}}: an interface entry is going
    to be removed from the interface table. This is a pre-delete
    notification so that clients have access to interface data that are
    possibly needed to react to the change
  \item \tbf{\ttt{NF\_INTERFACE\_CONFIG\_CHANGED}}: a configuration setting
    in an interface entry has changed (e.g. MTU or IP address)
  \item \tbf{\ttt{NF\_INTERFACE\_STATE\_CHANGED}}: a state variable in an
    interface entry has changed (e.g. the up/down flag)
\end{itemize}

In all those notifications, the data field is a pointer to the
corresponding \cppclass{InterfaceEntry} object. This is even true for
\ttt{NF\_INTERFACE\_DELETED} (which is actually a pre-delete notification).
\fi

\fi


%%% Local Variables:
%%% mode: latex
%%% TeX-master: "usman"
%%% End:

